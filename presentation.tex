\documentclass{beamer}
\usetheme{Madrid}
\usecolortheme{dolphin}
\usepackage{listings}
\usepackage{xcolor}
\usepackage{graphicx}
\usepackage{hyperref}
% \usepackage{fontawesome} % Uncomment if you need FontAwesome icons

% Custom colors
\definecolor{tidalblue}{RGB}{41, 128, 185}
\definecolor{codegray}{RGB}{240, 240, 240}
\definecolor{codegreen}{RGB}{0, 128, 0}

% Code listing style
\lstdefinestyle{tidal}{
  backgroundcolor=\color{codegray},
  commentstyle=\color{codegreen},
  basicstyle=\ttfamily\small,
  breakatwhitespace=false,
  breaklines=true,
  captionpos=b,
  keepspaces=true,
  showspaces=false,
  showstringspaces=false,
  showtabs=false,
  tabsize=2
}
\lstset{style=tidal}  % Apply this style globally

\title{Music and Code: Introduction to Livecoding with TidalCycles}
\subtitle{Algorithmic Control of External Hardware \& DAWs}
\author{Lenny Foret}  % Replace with your actual name
\institute{Music Technology Workshop for Locust Projects}
\date{\today}

\begin{document}

\begin{frame}
\titlepage
\end{frame}

\begin{frame}{What is TidalCycles?}
\begin{columns}
\column{0.6\textwidth}
\begin{itemize}
    \item Live coding environment for creating algorithmic patterns
    \item Especially powerful for rhythmic and pattern-based music
    \item Created by Alex McLean as part of the Algorave movement
    \item Open source, written in Haskell
    \item Concise syntax for complex musical ideas
\end{itemize}

\column{0.4\textwidth}
% \begin{center}
% \includegraphics[width=0.8\linewidth]{tidalcycles_logo.png}
% \end{center}
\end{columns}
\end{frame}

% Additional slides would go here, with similar commented image placeholders


\begin{frame}{Seminar Overview}
\begin{itemize}
    \item What is TidalCycles?
    \item Producer-oriented approach
    \item Using TidalCycles with external instruments
    \item Pattern concepts and syntax
    \item MIDI integration
    \item Automation and control
    \item Workshop: Analyzing a live performance
    \item Hands-on exercises
\end{itemize}
\end{frame}


%----------------------------------------------------------------------------------------
% SECTION 1: Introduction
%----------------------------------------------------------------------------------------

\begin{frame}{What is TidalCycles?}
\begin{columns}
\column{0.6\textwidth}
\begin{itemize}
    \item Live coding environment for creating algorithmic patterns
    \item Especially powerful for rhythmic and pattern-based music
    \item Created by Alex McLean as part of the Algorave movement
    \item Open source, written in Haskell
    \item Concise syntax for complex musical ideas
\end{itemize}

\column{0.4\textwidth}
\begin{center}

\end{center}
\end{columns}
\end{frame}

\begin{frame}{The Conventional TidalCycles Approach}
\begin{columns}
\column{0.6\textwidth}
Traditional setup:
\begin{itemize}
    \item TidalCycles as pattern generator
    \item SuperCollider as the sound engine
    \item SuperDirt as synthesizer/sampler
    \item Self-contained ecosystem
\end{itemize}

\column{0.4\textwidth}
\begin{center}
% [Image placeholder: Standard TidalCycles architecture]
\end{center}
\end{columns}
\end{frame}

\begin{frame}{Our Producer-Oriented Approach}
\begin{columns}
\column{0.6\textwidth}
\begin{itemize}
    \item Use TidalCycles as a flexible controller
    \item Integrate with standard production workflows
    \item Control hardware synths via MIDI
    \item Automate DAW parameters
    \item Trigger external instruments (909, 303, Hydrasynth)
    \item Focus on pattern generation, not sound synthesis
\end{itemize}

\column{0.4\textwidth}

\end{columns}
\end{frame}

\begin{frame}[fragile]{Pattern Basics}
A simple patter to play MIDI notes
\begin{lstlisting}[style=tidal]
-- Simple pattern: Play C4, E4, G4, C5 in sequence

d1 $ note "c4 e4 g4 c5" # s "midi"
\end{lstlisting}
\end{frame}


\begin{frame}[fragile]{Tidal Timing: Cycles and Patterns}
\begin{itemize}
    \item Default: 1 cycle = 1 bar (adjustable with \texttt{setcps})
    \item Tempo expressed in cycles per second
    \item For 120 BPM: \texttt{setcps (120/60/4)} = 0.5 cycles per second
\end{itemize}

\begin{lstlisting}[style=tidal]
-- Set tempo to 126 BPM
setcps (126/60/4)

-- Simple quarter note pattern
d1 $ s "midi" # note "c4"

-- Eighth notes
d1 $ s "midi" # note "c4 e4"

-- 16th notes (a 4-note pattern at twice the speed)
d1 $ s "midi" # note "c4 e4 g4 c5*2"
\end{lstlisting}
\end{frame}


\begin{frame}[fragile]{Pattern Syntax}
\begin{itemize}
    \item Simple sequences: \texttt{"a b c d"}
    \item Grouping: \texttt{"[a b] c"}
    \item Repetition: \texttt{"a*3 b"}
    \item Alternation: \texttt{"<a b> c"}
    \item Rest/silence: \texttt{"a ~ b c"}
    \item Euclidean rhythms: \texttt{"c(3,8)"} (3 events spread over 8 steps)
\end{itemize}

\begin{lstlisting}[style=tidal]
-- Different pattern examples
d1 $ note "c4 [e4 g4] c5*2" # s "midi"
d2 $ note "c4(3,8)" # s "midi"
d3 $ note "<c4 e4> <g4 f4>" # s "midi"
\end{lstlisting}
\end{frame}


\begin{frame}[fragile]{Pattern Manipulation}
Key functions for transforming patterns:
\begin{columns}
\column{0.5\textwidth}
\begin{itemize}
    \item \texttt{rev} - reverse
    \item \texttt{slow} - stretch pattern
    \item \texttt{fast} - compress pattern
    \item \texttt{jux} - juxtapose transformations
    \item \texttt{every} - apply function every n cycles
\end{itemize}

\column{0.5\textwidth}
\begin{itemize}
    \item \texttt{shuffle} - reorder elements
    \item \texttt{palindrome} - forward then backward
    \item \texttt{chop} - cut into smaller parts
    \item \texttt{striate} - interleave segments
    \item \texttt{swingBy} - add swing feel
\end{itemize}
\end{columns}

\begin{lstlisting}[style=tidal]
-- Add swing to a MIDI pattern
d1 $ swingBy 0.08 8 $ note "c4 e4 g4 c5 e5 g5 e5 c5" # s "midi"
\end{lstlisting}
\end{frame}


%----------------------------------------------------------------------------------------
% SECTION 3: MIDI Integration
%----------------------------------------------------------------------------------------

\begin{frame}[fragile]{MIDI Integration with TidalCycles}
\begin{itemize}
    \item Connect to MIDI devices through SuperCollider
    \item Send MIDI notes, CCs, and program changes
    \item Each pattern can target different MIDI devices/channels
    \item Synchronize with external equipment via MIDI clock
\end{itemize}

\begin{lstlisting}[style=tidal]
-- SuperCollider MIDI setup example
MIDIClient.init;
~midiOut = MIDIOut.newByName("USB MIDI Interface", "Port 1");
~dirt.soundLibrary.addMIDI(\midi, ~midiOut);

-- In TidalCycles, use device "midi"
d1 $ note "c4 e4 g4 c5" # s "midi" 
   # midichan 0
\end{lstlisting}
\end{frame}

\begin{frame}[fragile]{Practical MIDI Setup}
\begin{itemize}
    \item Multiple MIDI destinations: 
\end{itemize}

\begin{lstlisting}[style=tidal]
-- SuperCollider setup with multiple destinations
~midiOut1 = MIDIOut.newByName("TR-909", "MIDI IN");
~midiOut2 = MIDIOut.newByName("TB-303", "MIDI IN");
~midiOut3 = MIDIOut.newByName("Hydrasynth", "MIDI IN");

-- Add these devices to SuperDirt with custom names
~dirt.soundLibrary.addMIDI(\drums, ~midiOut1);
~dirt.soundLibrary.addMIDI(\bass, ~midiOut2);
~dirt.soundLibrary.addMIDI(\lead, ~midiOut3);

-- In TidalCycles, use specific devices
d1 $ note "36 ~ 38 ~" # s "drums"  -- kick/snare on 909
d2 $ note "c2 [c2 eb2] f2 g2" # s "bass"  -- bassline on 303
d3 $ note "c4 e4 g4 c5" # s "lead"  -- melody on Hydra
\end{lstlisting}
\end{frame}

\begin{frame}[fragile]{Control Changes and Automation}
\begin{itemize}
    \item Send CC messages to control parameters
    \item Create evolving parameter patterns
    \item Control filter cutoff, resonance, envelopes, etc.
\end{itemize}

\begin{lstlisting}[style=tidal]
-- Control cutoff (CC 74) on a 303
d2 $ note "c2 [c2 eb2] f2 g2" # s "bass"
   # ccn 74 # ccv (range 30 100 $ slow 8 $ sine)

-- Multiple CC values simultaneously
d3 $ note "c4 e4 g4 c5" # s "lead"
   # stack [
     ccn 74 # ccv (range 30 100 $ slow 8 $ sine),
     ccn 71 # ccv (range 10 90 $ slow 7 $ saw)
   ]
\end{lstlisting}
\end{frame}


\begin{frame}[fragile]{DAW Integration via IAC Bus}
\begin{itemize}
    \item Use IAC (Inter-Application Communication) Driver buses
    \item Route MIDI from TidalCycles to DAW channels
    \item Trigger instruments or control parameters in Ableton, Logic, etc.
\end{itemize}

\begin{lstlisting}[style=tidal]
-- In SuperCollider
~midiDAW = MIDIOut.newByName("IAC Driver", "Bus 1");
~dirt.soundLibrary.addMIDI(\daw, ~midiDAW);

-- In TidalCycles - control Ableton tracks
d1 $ note "c4 e4 g4 c5" # s "daw" # midichan 0  -- Track 1
d2 $ note "36 ~ 38 ~" # s "daw" # midichan 1    -- Track 2
d3 $ ccn 74 # ccv (range 0 127 $ slow 4 $ sine) 
   # s "daw" # midichan 2                       -- Automate Track 3
\end{lstlisting}
\end{frame}


%----------------------------------------------------------------------------------------
% SECTION 4: Advanced Pattern Techniques
%----------------------------------------------------------------------------------------

\begin{frame}[fragile]{Working with Scales and Chords}
\begin{itemize}
    \item Use musical scales rather than raw MIDI notes
    \item Build chord progressions with scale degrees
    \item Transpose patterns while staying in key
\end{itemize}

\begin{lstlisting}[style=tidal]
-- Using the scale function for melodies
d1 $ n (scale "minor" "0 2 4 7") # s "lead" |+ n "c"

-- Creating a chord pattern in F minor
d2 $ n (scale "minor" "[0,2,4]") # s "lead" |+ n "f"

-- Add transposition within the scale
d3 $ n (scale "minor" "[0,2,4]") |+ "-4" # s "lead" |+ n "f"
\end{lstlisting}
\end{frame}

\begin{frame}[fragile]{Complex Chord Example from Desert Minimal}
Breaking down a complex chord pattern:

\begin{lstlisting}[style=tidal]
-- Final complex chord pattern
d7 $ n (scale "minor" ( "[0,2,4, -14, -4 _ ]" 
  |+ "-4" 
  |+ "0")) 
  # s "seven" 
  |+ n ("0" |* 12) 
  |+ n "f" 
\end{lstlisting}

Components:
\begin{itemize}
    \item \texttt{scale "minor"} - Use minor scale
    \item \texttt{[0,2,4, -14, -4 \_]} - Chord structure with extensions
    \item \texttt{|+ "-4"} - Transposition down 4 scale degrees
    \item \texttt{|+ "0"} - Secondary transposition layer (can be changed)
    \item \texttt{|+ n ("0" |* 12)} - Octave setting
    \item \texttt{|+ n "f"} - Key of F
\end{itemize}
\end{frame}



\begin{frame}[fragile]{Advanced Pattern Structures}
\begin{itemize}
    \item Combining pattern operators
    \item Creating evolving patterns with slow transformations
    \item Using randomization and chaos functions
\end{itemize}

\begin{lstlisting}[style=tidal]
-- Complex pattern with multiple transformations
d1 $ every 4 (fast 2) $ every 3 (rev) 
   $ note (scale "minor" $ "0 [2 4] 7 <3 5>") 
   # s "lead" |+ n "f"

-- Pattern with evolving parameters
d2 $ note "c2 [~ c2] <eb2 f2> g2" # s "bass"
   # cutoff (range 300 2000 $ slow 8 $ sine)
   # resonance (range 0.1 0.8 $ slow 7 $ tri)
   # sustain (range 0.1 0.4 $ rand)
\end{lstlisting}
\end{frame}


%----------------------------------------------------------------------------------------
% SECTION 5: Case Study - Desert Minimal
%----------------------------------------------------------------------------------------

\begin{frame}{Case Study: Desert Minimal / Time Lapse}
\begin{itemize}
    \item Track: Time Lapse by Lenny Foret
    \item From the physical tape release
    \item Created with TidalCycles controlling external gear
    \item Source: \url{https://mmxximnml.bandcamp.com/track/lenny-foret-time-lapse}
\end{itemize}
\end{frame}


\begin{frame}[fragile]{Track Structure Breakdown}
\begin{enumerate}
    \item Intro: Atmospheric risers
    \item Section 1: Chord progression with complex harmony
    \item Section 2: Basic beat with foundational rhythm
    \item Section 3: Added textural elements
    \item Section 4-18: Various pattern and effect variations
    \item Final section: Sample slicing techniques
\end{enumerate}

We'll examine how different elements are controlled via MIDI and automation.
\end{frame}

\begin{frame}[fragile]{Chord Progression Analysis}
From the desert-minimal.tidal file:

\begin{lstlisting}[style=tidal]
d7 $ n (scale "minor" ( "[0,2,4, -14, -4 _ ]" 
  |+ "-4" 
  |+ "0")) 
  # s "seven" 
  |+ n ("0" |* 12) 
  |+ n "f" 
\end{lstlisting}

\begin{itemize}
    \item Uses a complex chord structure in F minor
    \item Includes extensions for rich harmony
    \item Channel "seven" represents an external synth instrument
    \item Forms the harmonic foundation of the track
\end{itemize}
\end{frame}

\begin{frame}[fragile]{Rhythm Section Analysis}
From the desert-minimal.tidal file:

\begin{lstlisting}[style=tidal]
d2 $ n ("{ 0 1 2 3 4 5 6 7 8 9 10 11 12 13 14 15}%8" 	
  |+ ("<0>" |* "16")
  |+ "-24")
  # s "drums" 
 
d3 $ n ("{2 ~ }%8" 	
  |+ ("<0>" |* "16")
  |+ "-24")
  # s "three" 
\end{lstlisting}

\begin{itemize}
    \item \texttt{drums} channel controls drum machine via MIDI
    \item Pattern uses 16 steps at 8 steps per cycle
    \item Transposition \texttt{|+ "-24"} for proper drum mapping
    \item Channel \texttt{three} adds sparse percussive elements
\end{itemize}
\end{frame}

\begin{frame}[fragile]{Effect Processing Analysis}
Control of external effects:

\begin{lstlisting}[style=tidal]
f3 $ segment 5 $ s "ch3"  
    # stack [lowcut 0,
          highcut 127,
          senda 10,
          sendb "<0 10 0 20 0 30>"]
\end{lstlisting}

\begin{itemize}
    \item \texttt{f3} controls effect parameters via MIDI CCs
    \item \texttt{segment 5} divides cycle into 5 equal parts
    \item \texttt{stack} applies multiple parameters simultaneously
    \item \texttt{senda}/\texttt{sendb} control external effect sends (reverb/delay)
    \item Pattern \texttt{"<0 10 0 20 0 30>"} creates rhythmic modulation
\end{itemize}
\end{frame}

\begin{frame}[fragile]{Adding Variation and Movement}
Evolution techniques:

\begin{lstlisting}[style=tidal]
-- Add alternating values to the chord pattern
d7 $ n (scale "minor" ( "[0,2,4, -14, -4 _ , ~ <6 2 >]" 
  |+ "-4" 
  |+ "<0>"))
  # s "seven" 
  |+ n ("0" |* 12) 
  |+ n "f" 

-- Add swing to drum patterns
d2 $ swingBy 0.08 8 $ n ("{ ~ 0 1 2 }%16" 	
  |+ ("<0 1>" |* "16")
  |+ "-24")
  # s "drums" 
\end{lstlisting}

\begin{itemize}
    \item \texttt{<6 2>} alternates values on successive cycles
    \item \texttt{swingBy 0.08 8} adds human-like timing variation
    \item These techniques create organic movement in machine patterns
\end{itemize}
\end{frame}

\begin{frame}[fragile]{Layer Integration for Intensity}
Creating builds and drops:

\begin{lstlisting}[style=tidal]
do
 f4 $ segment 5 $ s "ch4"  
     # stack [lowcut 0,
           highcut 100,
           senda 100,
           sendb 80]
 d2 $ n ("{ 0 1 2 3 4 5 6 7 8 9 10 11 12 13 14 15}%8" 
  |+ ("<0>" |* "16")
  |+ "-24")
  # s "drums" 
 d7 $ n (scale "minPent" ( "[0, <5 -2>, <8 2 1 0>, <-8 -6 -4>]" 
  |+ "-4" 
  |+ "0"))
  # s "seven" 
  |+ n ("0" |* 12) 
  |+ n "f" 
\end{lstlisting}

\begin{itemize}
    \item \texttt{do} block combines multiple changes simultaneously
    \item Increases intensity through coordinated parameter changes
    \item Builds tension through multiple evolving elements
\end{itemize}
\end{frame}


%----------------------------------------------------------------------------------------
% SECTION 6: Practical Implementation
%----------------------------------------------------------------------------------------

\begin{frame}{Hardware Setup for TidalCycles Production}
Recommended hardware configuration:
\begin{itemize}
    \item Computer running TidalCycles
    \item MIDI interface with multiple outputs
    \item Hardware synthesizers (analog/digital)
    \item Drum machines
    \item Effects processors
    \item Mixing console or audio interface
    \item Optional: DAW for recording and additional processing
\end{itemize}
\end{frame}

\begin{frame}[fragile]{Syncing with DAW and Hardware}
Options for synchronization:
\begin{itemize}
    \item MIDI clock from TidalCycles to devices
    \item MIDI clock from DAW to TidalCycles
    \item Ableton Link integration:
\end{itemize}
\begin{lstlisting}[style=tidal]
-- In SuperCollider setup file
-- Create Link connection with 2 beats per cycle
~link = LinkClock(2).latency_(s.latency);
    
-- Register the clock with SuperDirt
~link.tempo_(120/60/2); // Set tempo (120 BPM)
~dirt.serverControls = ~dirt.serverControls.addFirst(~link);
\end{lstlisting}
TidalCycles can adapt to different sync scenarios depending on your workflow.
\end{frame}

\begin{frame}[fragile]{Recording and Performance Setup}
Best practices:
\begin{itemize}
    \item Set up automation recording in your DAW
    \item Create templates for different hardware configurations
    \item Organize MIDI channels consistently
    \item Consider partial recording: MIDI patterns in DAW, but keep live control
    \item For performance: create modular patterns that can be mixed/matched
    \item Develop transitions between sections
\end{itemize}
\end{frame}

%----------------------------------------------------------------------------------------
% SECTION 7: Hands-on Exercises
%----------------------------------------------------------------------------------------

\begin{frame}[fragile]{Exercise 1: Building a Basic Pattern}
Create a simple pattern that controls an external synth:

\begin{lstlisting}[style=tidal]
-- Set the tempo
setcps (120/60/4)

-- Create a basic pattern
d1 $ note "c3 [~ c3] e3 g3" # s "bass"

-- Add some pattern variation
d1 $ every 4 (rev) $ note "c3 [~ c3] e3 g3" # s "bass"

-- Add parameter modulation
d1 $ every 4 (rev) $ note "c3 [~ c3] e3 g3" # s "bass"
   # ccn 74 # ccv (range 30 100 $ slow 4 $ sine)
\end{lstlisting}
\end{frame}

\begin{frame}[fragile]{Exercise 2: Creating a Drum Pattern}
Create drum patterns for an external drum machine:

\begin{lstlisting}[style=tidal]
-- Basic kick and snare pattern
d2 $ n "36 ~ 38 ~" # s "drums"

-- Add hi-hats
d3 $ n "~ 42 ~ 42" # s "drums"

-- Add a bit of swing
d2 $ swingBy 0.06 8 $ n "36 ~ 38 ~" # s "drums"
d3 $ swingBy 0.06 8 $ n "~ 42 ~ 42" # s "drums"

-- Create fill variations
d2 $ every 4 (fast 2) $ swingBy 0.06 8 $ n "36 ~ 38 ~" # s "drums"
\end{lstlisting}
\end{frame}

\begin{frame}[fragile]{Exercise 3: Chord Progressions}
Build a chord progression using scale degrees:

\begin{lstlisting}[style=tidal]
-- Basic triad in C minor
d4 $ n (scale "minor" "[0,2,4]") # s "pad" |+ n "c"

-- Create a progression
d4 $ n (scale "minor" "<[0,2,4] [3,5,7] [-2,0,2] [1,3,5]>") 
   # s "pad" |+ n "c"

-- Add a parameter sweep
d4 $ n (scale "minor" "<[0,2,4] [3,5,7] [-2,0,2] [1,3,5]>") 
   # s "pad" |+ n "c"
   # ccn 74 # ccv (range 20 110 $ slow 16 $ sine)
\end{lstlisting}
\end{frame}

\begin{frame}[fragile]{Exercise 4: Putting It All Together}
Combine elements into a coordinated pattern:

\begin{lstlisting}[style=tidal]
-- Use a "do" block to start multiple patterns
do
  -- Reset everything
  hush
  -- Set tempo
  setcps (120/60/4)
  -- Bass pattern
  d1 $ note "c3 [~ c3] e3 g3" # s "bass"
     # ccn 74 # ccv (range 30 100 $ slow 4 $ sine)
  -- Drum pattern
  d2 $ swingBy 0.06 8 $ n "36 ~ 38 ~" # s "drums"
  d3 $ swingBy 0.06 8 $ n "~ 42 ~ 42" # s "drums"
  -- Chord pattern
  d4 $ n (scale "minor" "<[0,2,4] [3,5,7] [-2,0,2] [1,3,5]>") 
     # s "pad" |+ n "c"
\end{lstlisting}
\end{frame}

%----------------------------------------------------------------------------------------
% SECTION 8: Conclusion and Resources
%----------------------------------------------------------------------------------------

\begin{frame}{Conclusion}
\begin{itemize}
    \item TidalCycles provides powerful pattern control for producers
    \item Integration with external gear preserves your sound while adding algorithmic complexity
    \item Benefits of this approach:
    \begin{itemize}
        \item Keep your familiar sounds and gear
        \item Add algorithmic complexity to traditional setups
        \item Combine programming precision with hardware character
        \item Create patterns beyond traditional sequencing
    \end{itemize}
    \item Start small: add TidalCycles to one aspect of your production
    \item Build up complexity as you become comfortable
\end{itemize}
\end{frame}

\begin{frame}{Resources}
\begin{itemize}
    \item Official TidalCycles documentation: \url{https://tidalcycles.org/docs/}
    \item Tidal Club community: \url{https://club.tidalcycles.org/}
    \item MIDI-specific tutorials: \url{https://tidalcycles.org/docs/pattern-language/midi/}
    \item Algorave community: \url{https://algorave.com/}
    \item Books:
    \begin{itemize}
        \item "Algorithmic Composition: A Guide to Composing Music with Nyquist" - Heintz
        \item "The SuperCollider Book" - Wilson, Cottle, Collins
    \end{itemize}
    \item YouTube channels with TidalCycles tutorials
    \item GitHub repositories with example patterns
\end{itemize}
\end{frame}

\end{document}
